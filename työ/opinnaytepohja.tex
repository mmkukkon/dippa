%%%%%%%%%%%%%%%%%%%%%%%%%%%%%%%%%%%%%%%%%%%%%%%%%%%%%%%%%%%%%%%%%%%%
%%%%%%%%%%%%%%%%%%%%%%%%%%%%%%%%%%%%%%%%%%%%%%%%%%%%%%%%%%%%%%%%%%%%
%%                                                                %%
%% Esimerkki opinnäytteen tekemisestä LaTeX:lla 20130926          %%
%% Alkuperäinen versio Luis Costa,  muutokset Perttu Puska        %%
%%                                                                %%
%% Tähän esimerkkiin kuuluu tiedostot                             %%
%%               opinnaytepohja.tex (versio 1.7)                  %%
%%               aaltothesis.sty (versio 1.7)                     %%
%%               kuva1.eps                                        %%
%%               kuva2.eps                                        %%
%%                                                                %%
%%                                                                %%
%% Kääntäminen                                                    %%
%% latex:                                                         %%
%%             $ latex opinnaytepohja                             %%
%%             $ latex opinnaytepohja                             %%
%%                                                                %%
%%   Tuloksena on tiedosto opinnayte.dvi, joka                    %%
%%   muutetaan ps-muotoon seuraavasti                             %%
%%                                                                %%
%%             $ dvips opinnaytepohja -o                          %%
%%                                                                %%
%% Selittävät kommentit on tässä esimerkissä varustettu           %%
%% %%-merkeillä ja muutokset, joita käyttäjä voi tehdä,           %%
%% on varustettu %-merkeillä                                      %%
%%                                                                %%
%%%%%%%%%%%%%%%%%%%%%%%%%%%%%%%%%%%%%%%%%%%%%%%%%%%%%%%%%%%%%%%%%%%%
%%%%%%%%%%%%%%%%%%%%%%%%%%%%%%%%%%%%%%%%%%%%%%%%%%%%%%%%%%%%%%%%%%%%


%% Käytä näitä, jos kirjoitat englanniksi
%%
%% Uncomment one of these if you write in English
%\documentclass[english,12pt,a4paper,pdftex]{article}
\documentclass[english,12pt,a4paper,dvips]{article}

%%
%% This package is required
%% Choose your school from arts, biz, chem, elec, eng, sci.
%% Choose the character encoding scheme used by your editor: utf8, latin1
\usepackage[elec,utf8]{aaltothesis} % this is the default in aaltothesis.sty
%\usepackage[elec,latin1]{aaltothesis}

%% Jos käytät latex-komentoa käännettäessä (oletusarvo), 
%% kuvat kannattaa tehdä eps-muotoon. älä käytä ps-muotoisia kuvia!
%% Käytä seuraavaa latex-komennon ja eps-kuvien kanssa 
%%
%% Jos taas käytät pdflatex-komentoa, joka kääntää tekstin suoraan
%% pdf-tiedostoksi, kuvasi on oltava jpg-formaatissa tai pdf-formaatissa.
%%
%% Use this if you run pdflatex and use jpg/pdf-format pictures.
%%
\usepackage{graphicx}

%% Jos et jostain syystä pidä, miten alla oleva hyperref-paketti käyttää
%% fontteja, värejä yms., käytä tämän paketin makroja muuttamaan
%% fonttimäärittelyt. Katso paketin dokumentaatiota. Paketti määrittelee
%% \url-makron, joten ota paketti käyttään, jos et käytä hyperref-pakettia.
%%
%% Use the macros in this package to change how the hyperref package below 
%% typesets its hypertext -- hyperlink colour, font, etc. See the package
%% documentation. It also defines the \url macro, so use the package when 
%% not using the hyperref package.
%\usepackage{url}

%% Saat pdf-tiedoston viittaukset ja linkit kuntoon seuraavalla paketilla.
%% Paketti toimii erityisen hyvin pdflatexin kanssa. 
%%
%% Use this if you want to get links and nice output with pdflatex
\usepackage[pdfpagemode=None,colorlinks=true,urlcolor=red,%
linkcolor=blue,citecolor=black,pdfstartview=FitH]{hyperref}

%% Matematiikan fontteja, symboleja ja muotoiluja lisää, näitä tarvitaan usein 
%%
%% Use this if you write hard core mathematics, these are usually needed
\usepackage{amsfonts,amssymb,amsbsy}  

%% Vaakasuunnan mitat, äLä KOSKE!
\setlength{\hoffset}{-1in}
\setlength{\oddsidemargin}{35mm}
\setlength{\evensidemargin}{25mm}
\setlength{\textwidth}{15cm}
%% Pystysuunnan mitat, äLä KOSKE!
\setlength{\voffset}{-1in}
\setlength{\headsep}{7mm}
\setlength{\headheight}{1em}
\setlength{\topmargin}{25mm-\headheight-\headsep}
\setlength{\textheight}{23cm}


%% Kaikki mikä paperille tulostuu, on tämän jälkeen
%%
%% Output starts here
\begin{document}

%% Korjaa vastaamaan korkeakouluasi, jos automaattisesti asetettu nimi on 
%% virheellinen 
%%
%% Change the school field to describe your school if the autimatically 
%% set name is wrong
% \university{aalto University}{aalto-Yliopisto}
% \school{School of Electrical Engineering}{SähkäTekniikan korkeakoulu}

%% Vain kandityälle: Korjaa seuraavat vastaamaan koulutusohjelmaasi
%%
%% Only for B.Sc. thesis: Choose your degree programme. 
\degreeprogram{Electronics and electrical engineering}%
{Elektroniikka ja sähkötekniikka}
%%

%% Vain DI/M.Sc.- ja lisensiaatintyälle: valitse laitos, 
%% professuuri ja sen professuurikoodi. 
%%
%% Only for M.Sc. and Licentiate thesis: Choose your department,
%% professorship and professorship code. 
\department{Department of Biomedical Engineering and Computational Science}%
{Lääketieteellisen tekniikan ja laskennallisen tieteen laitos}
\professorship{Circuit theory}{Piiriteoria}
\code{S-55}
%%

%% Valitse yksi näistä kolmesta
%%
%% Choose one of these:
%\univdegree{BSc}
\univdegree{MSc}
%\univdegree{Lic}

%% Oma nimi
%%
%% Should be self explanatory...
\author{Matleena Kukkonen}

%% Opinnäytteen otsikko tulee vain tähän. älä tavuta otsikkoa ja
%% vältä liian pitkää otsikkotekstiä. Jos latex ryhmittelee otsikon
%% huonosti, voit joutua pakottamaan rivinvaihdon \\ kontrollimerkillä.
%% Muista että otsikkoja ei tavuteta! 
%% Jos otsikossa on ja-sana, se ei jää rivin viimeiseksi sanaksi 
%% vaan aloittaa uuden rivin.
%% 
%% Your thesis title. If the title is very long and the latex 
%% does unsatisfactory job of breaking the lines, you will have to
%% break the lines yourself with \\ control character. 
%% Do not hyphenate titles.
\thesistitle{Thesis template}{Opinnäyteohje}

\place{Espoo}
%% Kandidaatintyän päivämäärä on sen esityspäivämäärä! 
%% 
%% For B.Sc. thesis use the date when you present your thesis. 
\date{24.9.2013}

%% Kandidaattiseminaarin vastuuopettaja tai diplomityän valvoja.
%% Huomaa tittelissä "\" -merkki pisteen jälkeen, 
%% ennen välilyäntiä ja seuraavaa merkkijonoa. 
%% Näin tehdään, koska kyseessä ei ole lauseen loppu, jonka jälkeen tulee 
%% hieman pidempi väli vaan halutaan tavallinen väli.
%%
%% B.Sc. or M.Sc. thesis supervisor 
%% Note the "\" after the comma. This forces the following space to be 
%% a normal interword space, not the space that starts a new sentence. 
\supervisor{Prof.\ Risto Ilmoniemi}{Prof.\ Risto Ilmoniemi}

%% B.Sc. or M.Sc. thesis advisors(s). 
%%
%% Note that there has been a change in the official EN translation
%% of the Finnish title ``ohjaaja'' which in the previous version (1.5) 
%% of this document was called ``instructor''. The recommended
%% translation is now ``advisor''.  
%% However, the LaTeX internal variable remains \instructor
%% as there is little point to change the variable name. 
%%
\instructor{M.Sc. Tuomas Mutanen}{DI Tuomas Mutanen}

%% Aaltologo: syntaksi:
%% \uselogo{aaltoRed|aaltoBlue|aaltoYellow|aaltoGray|aaltoGrayScale}{?|!|''}
%% Logon kieli on sama kuin dokumentin kieli
%%
%% Aalto logo: syntax:
% \uselogo{aaltoRed|aaltoBlue|aaltoYellow|aaltoGray|aaltoGrayScale}{?|!|''}
%% Logo language is set to be the same as the document language.
\uselogo{aaltoBlue}{''}

%% Tehdään kansilehti
%%
%% Create the coverpage
\makecoverpage


%% Suomenkielinen tiivistelmä
%% 
%% Finnish abstract
%%
%% Tiivistelmän avainsanat
\keywords{Avainsanoiksi valitaan kirjoituksen sisältää keskeisesti kuvaavia
käsitteitä}
%% Tiivistelmän tekstiosa
\begin{abstractpage}[finnish]
  Tiivistelmässä on lyhyt selvitys (noin 100 sanaa)
  kirjoituksen tärkeimmästä sisällästä: mitä ja miten on tutkittu,
  sekä mitä tuloksia on saatu. 
\end{abstractpage}

%% Pakotetaan uusi sivu varmuuden vuoksi, jotta 
%% mahdollinen suomenkielinen ja englanninkielinen tiivistelmä
%% eivät tule vahingossakaan samalle sivulle
%%
%% Force new page so that English abstract starts from a new page
\newpage
%
%% English abstract, uncomment if you need one. 
%% 
%% Abstract keywords
\keywords{Resistor, Resistance,\\ Temperature}
%% Abstract text
\begin{abstractpage}[english]
 Your abstract in English. Try to keep the abstract short, approximately 
 100 words should be enough. Abstract explains your research topic, 
 the methods you have used, and the results you obtained.  
\end{abstractpage}
%% Note that 
%% if you are writting your master's thesis in English place the English
%% abstract first followed by the possible Finnish abstract

%% Esipuhe 
%%
%% Preface
\mysection{Preface}
Haluan kiittää Professori Pirjo 
Professoria ja ohjaajaani Olli Ohjaajaa hyvästä ja 
huonosta ohjauksesta.\\

\vspace{5cm}
Otaniemi, 24.9.2013

\vspace{5mm}
{\hfill Teemu T.\ A.\ Teekkari \hspace{1cm}}

%% Pakotetaan varmuuden vuoksi esipuheen jälkeinen osa
%% alkamaan uudelta sivulta
%%
%% Force new page after preface
\newpage


%% Sisällysluettelo
%% 
%% Table of contents. 
\thesistableofcontents


%% Symbols and abbreviations

\mysection{Symbols and abbreviations}

\subsection*{Symbols}

\begin{tabular}{ll}
%$|a_{ij}|^2$, $|a_i|^2$ & probability of two electrons having momenta
%    $\boldsymbol p_i$ and $\boldsymbol p_j$ ($\boldsymbol p_i$ for $|a_i|^2$) \\
%                 & at any given instant \\
$\mathbf{B}$  & magneettivuon tiheys  \\
$c$              & valon nopeus tyhjässä $\approx 3\times10^8$ [m/s]\\
%$p$              & magnitude of momentum \\
%$\boldsymbol p$, $\boldsymbol p_i$, $\boldsymbol p_i^{'}$  & momentum vector \\
%$p$              & magnitude of momentum \\
%$\boldsymbol p$, $\boldsymbol p_i$, $\boldsymbol p_i^{'}$  & momentum vector \\
%$\boldsymbol P$  &  \\
%$p_{\mathrm{F}}$ & Fermi momentum \\
$\omega_{\mathrm{D}}$    & Debye-taajuus \\
$\omega_{\mathrm{latt}}$ & hilan keskimääräinen fononitaajuus \\
$\uparrow$       & elektronin spinin suunta yläspäin\\
$\downarrow$     & elektronin spinin suunta alaspäin
\end{tabular}


\subsection*{Opetators}

\begin{tabular}{ll}
$\nabla \times \mathbf{A}$              & vektorin $\mathbf{A}$ roottori\\
$\displaystyle\frac{\mbox{d}}{\mbox{d} t}$ & derivaatta muuttujan $t$ suhteen\\
[3mm]
$\displaystyle\frac{\partial}{\partial t}$  & osittaisderivaatta muuttujan $t$ suhteen \\[3mm]
$\sum_i $                       & Summa indeksin $i$ yli\\
$\mathbf{A} \cdot \mathbf{B}$    & vektorien $\mathbf{A}$ ja $\mathbf{B}$ pistetulo
\end{tabular}

\subsection*{Lyhenteet}
%\subsection*{Abbreviations}

\begin{tabular}{ll}
AC         & vaihtovirta \\
APLAC      & an object-oriented analog circuit simulator and design tool \\
           & (originally Analysis Program for Linear Active Circuits) \\
BCS        & Bardeen-Cooper-Schrieffer \\ %% tavuviiva - nimien välissä 
DC         & tasavirta \\
TEM        & transverse eletromagnetic
\end{tabular}


%% Sivulaskurin viilausta opinnäytteen vaatimusten mukaan:
%% Aloitetaan sivunumerointi arabialaisilla numeroilla (ja jätetään
%% leipätekstin ensimmäinen sivu tyhjäksi, 
%% ks. alla \thispagestyle{empty}).
%% Pakotetaan lisäksi ensimmäinen varsinainen tekstisivu alkamaan 
%% uudelta sivulta clearpage-komennolla. 
%% clearpage on melkein samanlainen kuin newpage, mutta 
%% flushaa myäs LaTeX:n floatit 
%% 
%% Corrects the page numbering, there is no need to change these
\cleardoublepage
\storeinipagenumber
\pagenumbering{arabic}
\setcounter{page}{1}



\section{Introduction}

%% Leave first page empty
\thispagestyle{empty}

When brain is activated, the potential changes in neurons cause electical currents. In transcranial magnetic stimulation, these currents are generated artificially.  


%% In a thesis, every section starts a new page, hence \clearpage
\clearpage

%\section{Aikaisempi tutkimus}
\section{Background}
In TMS, a strong electric pulse is driven through a coil. Current pulse induces a magnetic field according to Biot-Savart law:
\begin{equation}
B(r,\textit{t}) = \frac{\mu_0}{4\pi}I(t) \oint_C \frac{dl(r^{\prime}) \times (r - r^{\prime})}{\mid r-r^{\prime} \mid ^{3}},
\end{equation}
where $B(r,t)$ is the strenght of magnetic field in location r, $dl(r^{\prime})$ is a differential element of a wire and $\mu_0$ is the permeability of vacuum. The changing magnetic field on the other hand induces an electric field to the brain. The induced electric field is induced according the Faradays law:
\begin{equation}
\nabla \times \bf{E} = - \frac{\partial \bf{B}}{\partial t}
\end{equation}
Currents of electric field may exite neurons in the brain and cause neurons firing action potentials. The action potentials 

\subsection{TMS and TES affecting the brain}
\subsection{Previous studies}
\subsection{}

\clearpage

\section{Methods}
\subsection{Data}
\subsection{Experimental paradigm}
\subsection{Data analysis}

\clearpage

\section{Results}
%\section{Results}


\clearpage

\section{Discussion}
%\section{Summary} 

\clearpage
%% Lähdeluettelo
%%
%% \phantomsection varmistaa, että hyperref-paketti latoo hypertekstilinkit
%% oikein.
%%
%% The \phantomsection command is nessesary for hyperref to jump to the 
%% correct page, in other words it puts a hyper marker on the page.

\phantomsection
\addcontentsline{toc}{section}{References}
\begin{thebibliography}{99}

%% Alla pilkun jälkeen on pakotettu oikea väli \<välilyänti>-merkeillä.
\bibitem{Kauranen} Kauranen,\ I., Mustakallio,\ M. ja Palmgren,\ V.
  \textit{Tutkimusraportin kirjoittamisen opas opinnäytetyön
    tekijäille.}  Espoo, Teknillinen korkeakoulu, 2006.

\bibitem{Itkonen} Itkonen,\ M. \textit{Typografian käsikirja.} 3.\
  painos.  Helsinki, RPS-yhtiät, 2007.

\bibitem{Koblitz} Koblitz,\ N. \textit{A Course in Number Theory and
    Cryptography. Graduate Texts in Mathematics 114.}  2.\ painos. New
  York, Springer, 1994.

%% Kun on useampi nimikirjain, jokaisen nimikirjaimen väliin
%% kuuluu välilyönti. Oikea välin määrä on saatu \<välilyönnillä>
\bibitem{bcs} Bardeen,\ J., Cooper,\ L.\ N. ja Schrieffer,\ J.\ R.
  Theory of Superconductivity. \textit{Physical Review,} 1957, vol.\
  108, nro~5, s.\ 1175--1204.

\bibitem{Deschamps} Deschamps,\ G.\ A. Electromagnetics and
  Differential Forms. \textit{Proceedings of the IEEE,} 1981, vol.\
  69, nro~6, s.\ 676--696.

%% Alla esimerkki englanninkielisen tavuttamisen pakottamisesta.
%% Oletusarvoisesti käytetään suomalaista tavutusta, mutta viitteissä
%% esiintyy usein muunkielisiä lauseita, jotka tulevat siten tavutetuksi
%% suomen kielen sääntäjen mukaan. Tämän voi korjata \foreignlanguage-
%% komennolla, jonka ensimmäinen parametri on vieraan kielen nimi ja toinen 
%% on vieraalla kielellä tavutettava teksti. 
\bibitem{Sihvola} Sihvola,\ A.\ et al.
  Interpretation of measurements of helix 
and bihelix superchiral structures. Teoksessa: Jacob,\ A.\ F. ja
  Reinert,\ J. (toim.) \textit{Bianisotropics '98 7th International
    Conference on Complex Media.}  Braunschweig, 3.--6.6.1998.
  Braunscweig, Technische Universität Braunschweig, 1998, s.\
  317--320.

%% Alla on suomalainen yhdistelmäsukunimi. Sen nimien välissä 
%% käytetään yhdysmerkkiä l. tavuviivaa, kirjoitetaan -.
\bibitem{Lindblom} Lindblom-Ylänne,\ S. ja Wager,\ M.  Tieteellisten
  opinnäytetäiden ohjaaminen. Teoksessa: Lindblom-Ylänne,\ S. ja
  Nevgi,\ A. (toim.) \textit{Yliopisto- ja korkeakouluopettajan
    käsikirja.}  Helsinki, WSOY, 2004, s.\ 314--325.
 
\bibitem{Miinusmaa} Miinusmaa,\ H. Neliskulmaisen reiän poraamisesta
  kolmikulmaisella poralla. Diplomityä, Teknillinen korkeakoulu,
  konetekniikan osasto, Espoo, 1977.

%% Tässä taas pakotettu englanninkielinen tavutus. 
%% Pedanttinen kirjoittaja pakottaa tietysti jokaiseen englanninkieliseen
%% lauseeseen englannin tavutuksen, mutta tässä esityksessä ei näin ole
%% tehty selvyyden ja lähdekoodin luettavuuden takia. 
\bibitem{Loh} Loh,\ N.\ C. High-Resolution Micromachined
  Interferometric Accelerometer. Master's Thesis, Massachusetts
  Institute of Technology, Cambridge,
  Massachusetts, 1992.

\bibitem{Lonnqvist} Lännqvist,\ A.
 Applications of hologram-based compact
    range: antenna radiation pattern, radar cross section, and
    absorber reflectivity measurements. Väitäskirja, Teknillinen
  korkeakoulu, sähkä- ja tietoliikennetekniikan osasto, 2006.

\bibitem{sfs} SFS 5342. Kirjallisuusviitteiden laatiminen. 2.\ painos.
  Helsinki, Suomen standardisoimisliitto, 2004. 20~s.

\bibitem{haastattelu} Palmgren,\ V. Suunnittelija. Teknillinen
  korkeakoulu, kirjasto. Otaniementie 9, 02150 Espoo. Haastattelu
  15.1.2007.

\bibitem{Ribeiro} Ribeiro,\ C.\ B., Ollila,\ E. ja Koivunen,\ V.
  Stochastic Maximum-Likelihood Method for
    MIMO Propagation Parameter Estimation. \textit{IEEE Transactions
    on Signal Processing,} verkkolehti, vol.\ 55, nro~1, s.\ 46--55.
  Viitattu 19.1.2007. Lehti ilmestyy myäs painettuna. DOI:
  10.1109/TSP.2006.882057.

\bibitem{Stieber} Stieber,\ T. GnuPG Hacks. \textit{Linux Journal,}
  verkkolehti, 2006, maaliskuu, nro~143. Viitattu 19.1.2007. Lehti
  ilmestyy myäs painettuna. Saatavissa:
  \url{http://www.linuxjournal.com/article/8732.}

\bibitem{kone} Pohjois-Koivisto,\ T. Voiko kone tulevaisuudessa arvata
  tahtosi?  \textit{Apropos,} verkkolehti, helmikuu, nro~1, 2005.
  Viitattu 19.1.2007.  Saatavissa:
  \url{http://www.apropos.fi/1-2005/prima.php.}

\bibitem{Adida} Adida,\ B.  Advances in Cryptographic Voting Systems.
  Verkkodokumentti. Ph.D.\ Thesis, Massachusetts Institute of
  Technology, Cambridge, Massachusetts,
  2006. Viitattu 19.1.2007.  Saatavissa:
  \url{http://crypto.csail.mit.edu/~cis/theses/adida-phd.pdf.}

\bibitem{viittaaminen} Kilpeläinen,\ P. WWW-lähteisiin viittaaminen
  tutkielmatekstissä. Verkkodokumentti. Päivitetty 26.11.2001.
  Viitattu 19.1.2007. Saatavissa:
  \url{http://www.cs.uku.fi/~kilpelai/wwwlahteet.html.}

\end{thebibliography}

%% Liitteet
%% Appendices
\clearpage
\appendix
%\phantomsection
%%
%% Lisää tekstin "Liitteet" sisällysluetteloon
%%
%% Adds the word "Appendices" to the table of contents
%\addcontentsline{toc}{section}{Liiteet}
%\addcontentsline{toc}{section}{Appendices}

\section{Appendix example \label{AppendixA}}
%% Liitteiden kaavat, taulukot ja kuvat numeroidaan omana kokonaisuutenaan
%%
%% Equations, tables and figures have their own numbering in Appendices
\renewcommand{\theequation}{A\arabic{equation}}
\setcounter{equation}{0}  
\renewcommand{\thefigure}{A\arabic{figure}}
\setcounter{figure}{0}
\renewcommand{\thetable}{A\arabic{table}}
\setcounter{table}{0}

Liitteet eivät ole opinnäytteen kannalta välttämättämiä ja 
opinnäytteen tekijän on 
kirjoittamaan ryhtyessään hyvä ajatella pärjäävänsä ilman liitteitä.
Kokemattomat kirjoittajat, jotka ovat huolissaan
tekstiosan pituudesta, paisuttavat turhan 
helposti liitteitä pitääkseen tekstiosan pituuden annetuissa rajoissa.
Tällä tavalla ei synny hyvää opinnäytettä.   

Liite on itsenäinen kokonaisuus, vaikka se täydentääkin tekstiosaa.
Liite ei siten ole pelkkä listaus, kuva tai taulukko, vaan 
liitteessä selitetään aina sisällän laatu ja tarkoitus. 

Liitteeseen voi laittaa esimerkiksi listauksia. Alla on 
listausesimerkki tämän liitteen luomisesta. 

%% Verbatim-ympäristä ei muotoile tai tavuta tekstiä. Fontti on monospace.
%% Verbatim-ympäristän sisällä annettuja komentoja ei LaTeX käsittele. 
%% Vasta \end{verbatim}-komennon jälkeen jatketaan käsittelyä.
\begin{verbatim}
	\clearpage
	\appendix
	\addcontentsline{toc}{section}{Liite A}
	\section*{Liite A}
	...
	\thispagestyle{empty}
	...
	tekstiä
	...
	\clearpage
\end{verbatim}

Kaavojen numerointi muodostaa liitteissä oman kokonaisuutensa:
\begin{eqnarray}
d \wedge A  &=& F, \label{liitekaava1}\\
d \wedge F  &=& 0. \label{liitekaava2}
\end{eqnarray}


\clearpage
\section{Another appendix \label{AppendixB}}

%% Liitteiden kaavat, taulukot ja kuvat numeroidaan omana kokonaisuutenaan
%%
%% Equations, tables and figures have their own numbering in Appendices
\renewcommand{\theequation}{B\arabic{equation}}
\setcounter{equation}{0}  
\renewcommand{\thefigure}{B\arabic{figure}}
\setcounter{figure}{0}
\renewcommand{\thetable}{B\arabic{table}}
\setcounter{table}{0}

Liitteissä voi myäs olla kuvia, jotka
eivät sovi leipätekstin joukkoon:
%% Ympäristän figure parametrit htb pakottavat
%% kuvan tähän, eikä LaTeX yritä siirrellä niitä
%% hyväksi katsomaansa paikkaan. 
%% Ympäristää center voi käyttää \centering-
%% komennon sijaan
%%
%% Example of a figure, note the use of htb parameters which force
%% the figure to be inserted here
\begin{figure}[htb]
\begin{center}
\includegraphics[height=8cm]{kuva2}
\end{center}
\caption{Kuvateksti, jossa on liitteen numerointi \label{liitekuva}}
\end{figure}
%%
Liitteiden taulukoiden numerointi on kuvien ja kaavojen kaltainen:
\begin{table}[htb]
\caption{Taulukon kuvateksti. \label{liitetaulukko}}
\begin{center}
\fbox{
\begin{tabular}{lp{0.5\linewidth}}
9.00--9.55  & Käytettävyystestauksen tiedotustilaisuus (osanottajat
ovat saaneet sähkäpostitse valmistautumistehtävät, joten tiedotustilaisuus
voidaan pitää lyhyenä).\\
9.55--10.00 & Testausalueelle siirtyminen
\end{tabular}}
\end{center}
\end{table}
Kaavojen numerointi muodostaa liitteissä oman kokonaisuutensa:
\begin{eqnarray}
T_{ik} &=& -p g_{ik} + w u_i u_k + \tau_{ik},  \label{liitekaava3} \\
n_i    &=& n u_i + v_i.                        \label{liitekaava4}
\end{eqnarray}

\end{document}
